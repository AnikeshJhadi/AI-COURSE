\documentclass[a4paper,10pt]{article} 
\setlength{\parskip}{0pt}
\setlength{\parindent}{0pt}
\setlength{\voffset}{-15pt}
\usepackage[a4paper, margin=2.5cm]{geometry} 
\usepackage[onehalfspacing]{setspace} 
\usepackage[T1]{fontenc} 
\usepackage[utf8]{inputenc} 
\usepackage{charter} 
\usepackage{microtype} 
\usepackage[english, ngerman]{babel} 
\usepackage{amsthm, amsmath, amssymb} 
\usepackage{marvosym, wasysym} 
\usepackage{float} 
\usepackage[final, colorlinks = true, 
            linkcolor = black, 
            citecolor = black,
            urlcolor = black]{hyperref} 
\usepackage{graphicx, multicol} 
\usepackage{xcolor} 
\usepackage{rotating} 
\usepackage{listings, style/lstlisting}
\usepackage{pseudocode} 
\usepackage{style/avm} 
\usepackage{booktabs} 
\usepackage{tikz-qtree} 
\tikzset{every tree node/.style={align=center,anchor=north},
         level distance=2cm} 
\usepackage{style/btree} 
\usepackage{titlesec} 
\renewcommand\thesection{\arabic{section}.} 
\titleformat{\section}{\large}{\thesection}{1em}{}
\renewcommand\thesubsection{\alph{subsection})} 
\titleformat{\subsection}{\large}{\thesubsection}{1em}{}
\renewcommand\thesubsubsection{\roman{subsubsection}.}
\titleformat{\subsubsection}{\large}{\thesubsubsection}{1em}{}
\usepackage[all]{nowidow} 
\usepackage[backend=biber,style=numeric,
            sorting=nyt, natbib=true]{biblatex}
\addbibresource{main.bib}
\usepackage{csquotes} 
\usepackage[yyyymmdd]{datetime} 
\renewcommand{\dateseparator}{-} 
\usepackage{fancyhdr} 
\pagestyle{fancy} 
\fancyhead{}\renewcommand{\headrulewidth}{0pt}
\fancyfoot[L]{\textsc{ModuleShorthand00}} 
\fancyfoot[C]{}
\fancyfoot[R]{\thepage} 
\newcommand{\note}[1]{\marginpar{\scriptsize \textcolor{red}{#1}}} 
\begin{document}
\title{template_assignment} 
\fancyhead[C]{}
\begin{minipage}{0.295\textwidth} 
\raggedright
NIT RAIPUR\\ 
\footnotesize 
Anikesh Jhadi,19111007 
\medskip\hrule
\end{minipage}
\begin{minipage}{0.4\textwidth} 
\centering 
\large 
Assignment 04\\ 
\normalsize 
AI in Graphology\\ 
\end{minipage}
\begin{minipage}{0.295\textwidth} 
\raggedleft
\today\\ 
\footnotesize 
jhadianikesh@gmail.com
\medskip\hrule
\end{minipage}
\section{Introduction}
Graphology is the analysis of the physical characteristics and patterns of handwriting with attempt to identify the writer, indicate the psychological state at the time of writing, or evaluate personality characteristics.and it is generally considered a pseudoscience or scientifically questionably practice.Graphology has been controversial for more than a century. Although supporters point to the anecdotal evidence of positive testimonials as a reason to use it for personality evaluation, empirical studies fail to show the validity claimed by its supporters.Although graphology had some support in the scientific community before the mid-twentieth century, more recent research rejects the validity of graphology as a tool to assess personality and job performance.Today it is considered to be a pseudoscience.Many studies have been conducted to assess its effectiveness to predict personality and job performance. Recent studies testing the validity of using handwriting for predicting personality traits and job performance have been consistently negative.

Other divining techniques like iridology, phrenology, palmistry, and astrology also have differing schools of thought, require years of training, offer expensive certifications, and fail just as soundly when put to a scientific controlled test. Handwriting analysis does have its plausible-sounding separation from those other techniques though, and that's the whole "handwriting is brainwriting" idea — traits from the brain will be manifested in the way that it controls the muscles of the hand. Unfortunately, this is just as unscientific as the others. No amount of sciencey sounding language can make up for a technique failing when put to a scientifically controlled test.

\section{Graphology in the age of AI}
We have reached an era, where machines have almost overtaken humans.Though AI is currently in its nascent stage, it is very much on its way to rule our lives.


Graphology (study of an individual’s handwriting patterns to help understand his or her personality/behaviour) is the most valuable tool which can work wonders. Graphology helps to reveal many tangible and intangible aspects of human personality which cannot be perceived at conscious level. It’s a total eye opener. Many interesting and unusual aspects of human persona can be revealed beforehand just in seconds, which machines may fail to detect. For instance, if an employee in a company is prone to dishonesty or has a suspicious nature, it can be efficiently divulged in his/her handwriting instantly, and so the employers can take appropriate actions well in advance and prevent potential losses to the company. It can be efficiently utilised in many sectors like Banking, all kinds of organizations, Recruitment, Educational Institutions, etc. It is very beneficial for scanning right candidates for a job, to select appropriate business partner /spouse, to know about illnesses, childhood trauma, to detect fraud and in every possible area where humans are involved. Moreover, necessary and positive changes can be made through Graphotherapy (making necessary changes in one’s thought process by making appropriate changes in handwriting pattern in order to gain positive results.)


The biggest advantage of Graphology is no matter how frequently/infrequently you write, or how good /bad your handwriting is, the moment you move your pen on the paper, it begins to reveal your thought process. It is just your willingness to write a handwriting sample to get your handwriting analysed that really matters.
\end{document}
