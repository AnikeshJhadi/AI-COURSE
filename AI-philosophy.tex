\documentclass[a4paper,10pt]{article} 
\setlength{\parskip}{0pt}
\setlength{\parindent}{0pt}
\setlength{\voffset}{-15pt}
\usepackage[a4paper, margin=2.5cm]{geometry} 
\usepackage[onehalfspacing]{setspace} 
\usepackage[T1]{fontenc} 
\usepackage[utf8]{inputenc} 
\usepackage{charter} 
\usepackage{microtype} 
\usepackage[english, ngerman]{babel} 
\usepackage{amsthm, amsmath, amssymb} 
\usepackage{marvosym, wasysym} 
\usepackage{float} 
\usepackage[final, colorlinks = true, 
            linkcolor = black, 
            citecolor = black,
            urlcolor = black]{hyperref} 
\usepackage{graphicx, multicol} 
\usepackage{xcolor} 
\usepackage{rotating} 
\usepackage{listings, style/lstlisting}
\usepackage{pseudocode} 
\usepackage{style/avm} 
\usepackage{booktabs} 
\usepackage{tikz-qtree} 
\tikzset{every tree node/.style={align=center,anchor=north},
         level distance=2cm} 
\usepackage{style/btree} 
\usepackage{titlesec} 
\renewcommand\thesection{\arabic{section}.} 
\titleformat{\section}{\large}{\thesection}{1em}{}
\renewcommand\thesubsection{\alph{subsection})} 
\titleformat{\subsection}{\large}{\thesubsection}{1em}{}
\renewcommand\thesubsubsection{\roman{subsubsection}.}
\titleformat{\subsubsection}{\large}{\thesubsubsection}{1em}{}
\usepackage[all]{nowidow} 
\usepackage[backend=biber,style=numeric,
            sorting=nyt, natbib=true]{biblatex}
\addbibresource{main.bib}
\usepackage{csquotes} 
\usepackage[yyyymmdd]{datetime} 
\renewcommand{\dateseparator}{-} 
\usepackage{fancyhdr} 
\pagestyle{fancy} 
\fancyhead{}\renewcommand{\headrulewidth}{0pt}
\fancyfoot[L]{\textsc{ModuleShorthand00}} 
\fancyfoot[C]{}
\fancyfoot[R]{\thepage} 
\newcommand{\note}[1]{\marginpar{\scriptsize \textcolor{red}{#1}}} 
\begin{document}
\title{template_assignment} 
\fancyhead[C]{}
\begin{minipage}{0.295\textwidth} 
\raggedright
NIT RAIPUR\\ 
\footnotesize 
Anikesh Jhadi,19111007 
\medskip\hrule
\end{minipage}
\begin{minipage}{0.4\textwidth} 
\centering 
\large 
Assignment 01\\ 
\normalsize 
Artificial Intelligence\\ 
\end{minipage}
\begin{minipage}{0.295\textwidth} 
\raggedleft
\today\\ 
\footnotesize 
jhadianikesh@gmail.com
\medskip\hrule
\end{minipage}
\section{Introduction}
 The philosophy of artificial intelligence is a branch of the philosophy of technology that explores artificial intelligence and its implications for knowledge and understanding of intelligence, ethics, consciousness, epistemology, and free will.\\
\textbf {Some important propositions in philosophy of AI :- }\\
\textbf {Turing's "polite convention":} If a machine behaves as intelligently as a human being, then it is as intelligent as a human being. \\
\textbf {The Dartmouth proposal:} "Every aspect of learning or any other feature of intelligence can be so precisely described that a machine can be made to simulate it. \\
\textbf {Allen Newell and Herbert A. Simon's physical symbol system hypothes}: "A physical symbol system has the necessary and sufficient means of general intelligent action.\\
\textbf  {John Searle's strong AI hypothesis:}  "The appropriately programmed computer with the right inputs and outputs would thereby have a mind in exactly the same sense human beings have minds."\\
\textbf {Hobbes' mechanism: }"For 'reason' ... is nothing but 'reckoning,' that is adding and subtracting, of the consequences of general names agreed upon for the 'marking' and 'signifying' of our thoughts. \\
 \section{Can a machine display general intelligence ?}
Machines have a limit and they lack special part of human mind responsible for intelligence and it cannot be replicated.
However the Turing Child Machine proposal sidesteps the need for precise design-time description all together. 
\subsection{Intelligence} 
\textbf {Turing Test :-}
Turing reduced  intelligence into conversations, If the machine can answer any question into words used by person it is an intelligent as human. However there is criticism that just only if the machine uses human words doesn’t mean it is as intelligent as human. 
\subsection{Intelligent agent}
It is defined as if the machine can maximize the expected value by the help of past experiences and knowledge than it is intelligent.
\subsection{Arguments that a machine can display general intelligence}
Some say that human thinking is a symbol processing.These arguments show that human thinking does not consist (solely) of high level symbol manipulation. They do not show that artificial intelligence is impossible, only that more than symbol processing is required.\\
\section{Can a machine have a mind, consciousness, and mental states?}
Searle differentiated AI into Strong and Weak AI :- \\ 
Strong AI- A physical symbol that can have mind and mental states .\\
Weak AI – A physical system that can act intelligently. \\
However these statement do not answer our question of can machine have a mind ? \\
\subsection{Consciousness, Mind and Mental States }
Consciousness can be defined ranging from  energetic fluid that permeates life to self awareness to something we know on a daily basis. \\
\subsection{Arguments that Computer do not have mind }
\textbf {Searles Chinese Room - }The Chinese room argument holds that a digital computer executing a program cannot have a "mind", "understanding" or "consciousness", regardless of how intelligently or human-like the program may make the computer behave.\\
 \textbf {Leibniz' mill, Davis's telephone exchange, Block's Chinese nation and Blockhead} \\
\section{Is thinking a kind of computation?}
The computational theory of mind or "computationalism" claims that the relationship between mind and brain is similar. \\
our intelligence derives from a form of calculation, similar to arithmetic. This is the physical symbol system hypothesis discussed above, and it implies that artificial intelligence is possible
Mental states are just implementations of (the right) computer programs. \\
\subsection{There are other question which are debated upon on and they are necessary to be pondered. }
Can a machine have emotions,it be self aware,original or creative,imitate all human characteristics,have a soul?
\subsection{Can a machine be benevolent or hostile?}
There is the question of what will happen when the machines will achieve autonomy because it is argued machines will have millions times more intelligence in just a few years. Some question about the use of robots in military use because those robots are given a autonomy to some extent and not enough attention to their implications.
\section{Views on the role of philosophy}
Some scholars argue that the AI community's dismissal of philosophy is detrimental. Physicist David Deutsch argues that without an understanding of philosophy or its concepts, AI development would suffer from a lack of progress. \\

\textbf{Important keywords :- Knowledge,Intelligence,Epistemology,Ethics,Consciousness,Computation,
Freewill,Symbol processing,Robotics,Intelligence Agent }
\end{document}
